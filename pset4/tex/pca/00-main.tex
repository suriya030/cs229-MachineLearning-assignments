\item \points{15} {\bf PCA} 

In class, we showed that PCA finds the ``variance maximizing'' directions onto
which to project the data.  In this problem, we find another interpretation of PCA. 

Suppose we are given a set of points $\{x^{(1)},\ldots,x^{(\nexp)}\}$. Let us
assume that we have as usual preprocessed the data to have zero-mean and unit variance
in each coordinate.  For a given unit-length vector $u$, let $f_u(x)$ be the 
projection of point $x$ onto the direction given by $u$.  I.e., if 
${\cal V} = \{\alpha u : \alpha \in \Re\}$, then 
\[
f_u(x) = \arg \min_{v\in {\cal V}} ||x-v||^2.
\]





\begin{enumerate}
 	\item \subquestionpoints{10} Recall that the principal component is the largest eigenvalue of $X^\top X$ for data matrix $X \in R^{n \times d}$. Show that the unit-length vector $u$ that minimizes the 
mean squared error between projected points and original points corresponds
to the first principal component for the data. I.e., show that
$$ \arg \min_{u:u^Tu=1} \sum_{i=1}^\nexp \|x^{(i)}-f_u(x^{(i)})\|_2^2 \ .$$
gives the first principal component.


{\bf Remark.} If we are asked to find a $k$-dimensional subspace onto which to
project the data so as to minimize the sum of squares distance between the
original data and their projections, then we should choose the $k$-dimensional
subspace spanned by the first $k$ principal components of the data.  This problem
shows that this result holds for the case of $k=1$.
	\ifnum\solutions=1 {
	\begin{answer}

\end{answer}


  

} \fi

\item \subquestionpoints{5} Now we will explore the relationship between two of the most popular dimensionality reduction techniques, SVD and PCA, at a basic conceptual level. Before we proceed with the question itself, let us briefly recap the SVD and PCA techniques and a few important observations:

\begin{itemize}
    \item \textbf{Eigenvalue Decomposition}: First, recall that the eigenvalue decomposition of a real, symmetric, and square matrix \( B \) (of size \( d \times d \)) can be written as the following product:
    \[
    B = Q \Lambda Q^\top
    \]
    where \( \Lambda = \text{diag}(\lambda_1, \dots, \lambda_d) \) contains the eigenvalues of \( B \) (which are always real) along its main diagonal, and \( Q \) is an orthogonal matrix containing the eigenvectors of \( B \) as its columns.
    
    \item \textbf{Principal Component Analysis (PCA)}: Given a data matrix \( M \) (of size \( p \times q \)), we showed in part (a) that PCA involves finding eigenvectors of the matrix \( M^\top M \). The matrix of these eigenvectors can be thought of as a rigid rotation in a high-dimensional space. PCA then projects each row of $M$ onto the top $k$ principal components to produce a lower dimensional version of each data point (where $k << q$).
\end{itemize}

Now we turn to the question! Let us define a real matrix \( M \) (of size \( p \times q \)) and let us assume this matrix corresponds to a dataset with \( p \) data points and \( q \) dimensions.

\begin{enumerate}
    \item \subquestionpoints{2} Prove that \( M^\top M \) is real, symmetric, and square. Write its eigenvalue decomposition in terms of $Q, \Lambda, Q^T$.
    
    \item \subquestionpoints{2} SVD involves the decomposition of a data matrix \( M \in \mathbb{R}^{p \times q} \) into a product $M = U \Sigma V^\top$ where \( U \in \mathbb{R}^{p \times p} \) and \( V \in \mathbb{R}^{q \times q} \) are column-orthonormal matrices and \( \Sigma \in \mathbb{R}^{p \times q} \) is a diagonal matrix. The entries along the diagonal of \( \Sigma \) are referred to as the singular values of \( M \). Write a simplified expression for \( M^\top M \) in terms of \( V \), \( V^T \), and \( \Sigma \).\\
    \textit{Hint:} A matrix $A$ is column-orthonormal if and only if $A^T A = I$

    \item \subquestionpoints{1} What is the relationship (if any) between the eigenvalues of \( M^\top M \) and the singular values of \( M \)?

\end{enumerate}



\ifnum\solutions=1 {
	\begin{answer}

\end{answer}
} \fi


\end{enumerate}