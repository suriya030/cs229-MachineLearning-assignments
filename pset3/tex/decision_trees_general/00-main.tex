\item \points{30} {\bf Decision trees}

Consider the problem of predicting if a person has a college degree based on age and salary. Table~\ref{tab:decisionTree} contains training data for 10 individuals. 
\begin{table}[h!]
	\centering
	\begin{tabular}{c|c|c}
		\hline
		Age & Salary (\$1k) &  College degree\\\hline
		24& 40 &  Yes\\
		53& 52 &  No\\
		23&  25&  No\\
		25&  77&  Yes\\
		32&  48&  Yes\\
		52&  110&  Yes\\
		22&  38&  Yes\\
		43&  44&  No\\
		52&  27&  No\\
		48&  65& Yes\\ \hline
	\end{tabular}
	\caption{Training data for predicting college degree.}
	\label{tab:decisionTree}
\end{table}

For questions below, the answers may not be unique. Any plausible solution is acceptable. Keep two significant decimals in part (a) and (c).


\begin{enumerate}
    \item \subquestionpoints{3} Given a set of $n$ observations $(x_i, y_i)$ where $y_i$ is the label $y_i \in \{-1,1\}$, let $f_t(x)$ be the weak classifier at step $t$ and let $\hat{w}_t$ be its weight. First we note that the final classifier after $T$ steps is defined as
		\begin{align*}
			F(x) = \text{sign} \left\{\sum_{t=1}^T \hat{w}_t f_t(x) \right\}
			= \text{sign}\{f(x)\},
		\end{align*}
	where
	\begin{align*}
		f(x) = \sum_{t=1}^T \hat{w}_t f_t(x).
	\end{align*}
	We can assume that $f(x)$ is never exactly zero.

	Show that
	\begin{align*}
		\varepsilon_{\text{training}}
		:= \frac{1}{n} \sum_{i=1}^n 1_{\{F(x_i) \neq y_i\}}
		\le \frac{1}{n} \sum_{i=1}^n \exp(-f(x_i) y_i),
	\end{align*}
	where $1_{\{F(x_i) \neq y_i\}}$ is $1$ if $F(x_i) \neq y_i$ and $0$ otherwise.
    
	\ifnum\solutions=1 {
	\begin{answer}
Type your solutions here.
\end{answer}
        } \fi
        
 %    \input{decision_trees_general/05-implementation}
    
	% \ifnum\solutions=1 {
	% \input{decision_trees_general/05-implementation-sol}
 %        } \fi
    
    \item \subquestionpoints{4} Draw  a possible stump that we could select at the next boosting iteration. You need to draw both the decision boundary and its positive orientation. The answer may not be unique and any plausible solution is acceptable.
    
	\ifnum\solutions=1 {
	\begin{answer}{
Type your solutions here.	
}
\end{answer}
        } \fi
        
    \item \subquestionpoints{9}  We showed above that training error is bounded above by $\prod_{t=1}^T Z_t$. At step $t$ the values $Z_1$, $Z_2$, $\ldots$, $Z_{t-1}$ are already fixed therefore at step $t$ we can choose $\alpha_t$ to minimize $Z_t$. Let
		\begin{align*}
			\varepsilon_t 
			= \sum_{i=1}^n \alpha_{i, t} 1_{\{f_t(x_i) \neq y_i\}}
		\end{align*}
	be the weighted training error for the weak classifier $f_t(x)$. Then we can re-write the formula for $Z_t$ as
	\begin{align*}
		Z_t = (1-\varepsilon_t) \exp(-\hat{w}_t) + \varepsilon_t \exp(\hat{w}_t).
	\end{align*}
 \begin{enumerate}
 	\item [(i)] [3 points] First find the value of  $\hat{w}_t$ that minimizes $Z_t$. Then show that the corresponding optimal value is
	\begin{align*}
		Z^{\text{opt}}_t
		= 2 \sqrt{\varepsilon_t (1-\varepsilon_t)}.
	\end{align*}
	\item [(ii)] [3 points] Assume we choose $Z_t$ this way. Then re-write $\varepsilon_t = 1/2 - \gamma_t$, where $\gamma_t > 0$ implies better than random and $\gamma_t < 0$ implies worse than random. Then show that
	\begin{align*}
		Z_t \le \exp(-2 \gamma_t^2).
	\end{align*}
	(You may want to use the fact that $\log(1 - x) \le  -x$ for $0 \le  x < 1$.)
	\item [(iii)] [3 points] Finally, show that if each classifier is better than random, i.e.,  $\gamma_t > \gamma$ for all $t$ and $\gamma > 0$, then
	\begin{align*}
			\varepsilon_{\text{training}}
		\le \exp(-2T \gamma^2),
	\end{align*}
	which shows that the training error can be made arbitrarily small with enough steps.
 \end{enumerate}
    
	\ifnum\solutions=1 {
	\begin{answer}{
Type your solutions here.
}
\end{answer}
        } \fi
        
    \item \subquestionpoints{15} 
Now let’s implement a classification, univariate decision tree with misclassification loss (mentioned in Equation 1). The starter code is provided in 

\texttt{src/decision\_trees\_general/decision\_tree.py}. Fill in the functions marked with \texttt{\#TODO}. You are not allowed to use any package other than \texttt{NumPy}. You cannot assume there are only two classes. Report the accuracy output when running the Python script. For reference, the 
staff solution gives the same expected accuracy in part (a) for the college degree dataset (Table 1) 
and $93.33\%$ for the iris dataset.
    
	\ifnum\solutions=1 {
	\begin{answer}{
Type your solutions here.
}
\end{answer}
        } \fi
        
        
\end{enumerate}
