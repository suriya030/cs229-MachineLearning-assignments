
All students must submit an electronic PDF version of the written questions. We highly recommend typesetting your solutions via \LaTeX, and we will award one bonus point for typeset submissions. To edit on Overleaf, upload the \texttt{pset3.zip}  folder as new project. In the project view, click on the “Menu” button located on the top left corner. In the “Settings” subsection, you will need to set the "Main Document" to be "tex/ps3.tex". Also, make sure your \textbf{cursor is selected on ps3.tex}  before your press the \textbf{Recompile button}. Type your response in the \texttt{-sol.tex} file version for every problem. Inside \texttt{text/ps3.tex} file, you will need to change \texttt{\textbackslash def\textbackslash solutions\{0\}} to \texttt{\textbackslash def\textbackslash solutions\{1\}} and then re-compile. To export the PDF file, click the downward arrow next to the \textbf{Recompile button}.

\vspace{1mm}

All students must also submit a zip file of their source code to Gradescope, which should be created using the \texttt{make\_zip.py} script. You should make sure to (1) restrict yourself to only using libraries included in the \texttt{environment.yml} file, and (2) make sure your code runs without errors. Your submission may be evaluated by the auto-grader using a private test set, or used for verifying the outputs reported in the writeup. 


